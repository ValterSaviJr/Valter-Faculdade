\chapter{Abrigos de Cultivo}
Esta parte tratara sobre o funcionamento dos abrigos de cultivo, mostrar alguns tipos, porque utiliza-los, ócios encontrados no gerenciamento e técnicas utilizadas no controle dos mesmos
\section{sobre os abrigos}
Também conhecidos como estufas, os abrigos são estruturas altas, podendo ser de madeira, concreto, alumínio ou aço, cobertas por um material translúcido (filmes, placas, telados) assim protegendo o cultivo de ameaças externas como clima hostil (ventos, tempestades, chuva ou sol intensos) e melhor controle de pragas. dependendo do cultivo e solo, o produto pode ser cultivado tanto em bancadas ou no próprio solo.

Além da proteção fornecida, alguns abrigos podem  possuir ferramentas como ventiladores, sombrites, injetores de fertilizantes entre outros para assim poder criar um cultivo protegido, que traz mais benefícios, como redução na sazonalidade do produto, aumento na produção, exploração superior dos recursos de criação (nutrientes, luz solar e $Co^2$). Os modelo do abrigo depende muito da finalidade e a região do mesmo.


\afazer{alguma imagem ilustrando, quem sabe uma tabela simples que relacione alguns modelos de abrigos e tipos de cultivos}

\section{Alguns Modelos de Abrigos}

\section{Manejo do Ambiente}
Para que os benefícios citados anteriormente sejam alcançados, o agricultor deve estar atento as condições do abrigo, para utilizar de maneira responsável e eficiente os equipamentos dispostos, e para tal, "o mesmo deve conhecer muito bem as espécies a serem cultivas, principalmente quanto as exigências ambientais e nutricionais" (MANEJO DO AMBIENTE EM CULTIVO PROTEGIDO), pois caso o manejo não se apresente eficiente a qualidade do produto ou até mesmo a produção poderá ser comprometida. Devido a sensibilidade e complexidade do manejo torna o agricultor um 'refém' do abrigo, e também pelo conhecimento necessário do produto a ser cultivado falta mão de obra qualificada.

Tendo isso em vista alguns métodos de automação de abrigos foram estudados e implantados \afazer{será mesmo verdade? e quais são}



\afazer{técnicas de automação}
